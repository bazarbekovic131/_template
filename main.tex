\documentclass{report}
\input{preamble}
\input{macros}
\usepackage[utf8]{inputenc}
\begin{document}

%----------------------------------------------------------------------------------------
%	TITLE PAGE
%----------------------------------------------------------------------------------------

%\begingroup
%\thispagestyle{empty}
%\AddToShipoutPicture*{\put(0,-300){\includegraphics[scale=0.6]%{cover}}} % Image background
%\centering
%\vspace*{5cm}
%\endgroup

%----------------------------------------------------------------------------------------
\newpage
\pdfbookmark[section]{\contentsname}{toc}
\tableofcontents
\pagebreak

\chapter{Einstein's Relativity Theory}

The general Theory of Relativity describes the accelerated frames (Bezugssysteme) and therefore also the gravitation.

Bezugssystem is the part of coordinate systems that are at rest relative to each other. The frame where the Newton's axioms work is called the inertial frame (Inertialsystem). 

If some inertial frames are moving to each other, there cannot be conducted any mechanical experiment that allows us to determine which frame is moving and which one stays at rest and whether both of them are moving. This result is formulated by \emph{Newton's Relativity Principle}:
\thm{Newton'sches Relativitätsprinzip (Newton's Relativity Principle)}{The absolute movement can not be measured}

However, scientists began to question this theory from the 19. century, having a point that it can be indeed obtained by measuring the light speed.

\thm{Einstein'sche Postulate (Einstein's Postulates)}{
\begin{itemize}
    \item 1. Postulate: Absolute and uniform motion cannot be measured
    \item 2. Postulate: The speed of light doesn't depend on the motion state of the light source
    \item 2. Postulate (Alternative): Each observer misst the value for the light-speed in the vacuum.
\end{itemize}
}

\thm{Galilei-Transformation}{
\begin{equation}
\begin{aligned}
    x^{(a)} &= x^{(b)} + v^{(a)}_b \cdot t^{(b)}, && y^{(a)} &= y^{(b)}, \\
    z^{(a)} &= z^{(b)}, && t^{(a)} &= t^{(b)}
\end{aligned}
\label{eq:galileo}
\end{equation}

\begin{equation}
\begin{aligned}
    x^{(b)} &= x^{(a)} - v^{(a)}_b \cdot t^{(b)}, && y^{(b)} &= y^{(a)}, \\
    z^{(b)} &= z^{(a)}, && t^{(b)} &= t^{(a)}
\end{aligned}
\label{eq:galileo_inverse}
\end{equation}
}
\newpage
\thm{Lorentz-Transformation}{
\begin{equation}
\begin{aligned}
    x^{(a)} &= \gamma \left(x^{(b)} + v^{(a)}_b t^{(b)}\right), \\
    y^{(a)} &= y^{(b)}, \\
    z^{(a)} &= z^{(b)}, \\
    t^{(a)} &= \gamma \left(t^{(b)} + \frac{v^{(a)}_b x^{(b)}}{c^2}\right)
\end{aligned}
\label{eq:lorentz_transformation}
\end{equation}
}

\section{Time Dilation (Zeitdilatation)}
Betrachten wir zwei Ereignisse. In the Frame $S_b$ in two different time points $t_1^{(b)} \ \textrm{and} \ t_2^{(b)}$ on the point $x_0^{(b)}$. With the help of Lorentz-Transformation we can confirm the times $t_1^{(a)}$ and $t_2^{(a)}$ are:
\begin{equation}
\begin{aligned}
    t_1^A &= \gamma \left(t_1^B + \frac{v_B^A x_0^B}{c^2}\right) \\
    t_2^A &= \gamma \left(t_2^B + \frac{v_B^A x_0^B}{c^2}\right)
\end{aligned}
\end{equation}

The interval $t_2^A - t_1^A$ is equal to the Eigenzeit \emph{$\Delta t_{eigen}$}. The corresponding time is then $\Delta t = \gamma \Delta t_{eigen}$.

Time dilation can be obtained directly from Einstein's Postulates. For the observer standing in distance d from the mirror, time interval is between send of light beam and its return due to reflection on the mirror. All is happening the frame $S_B$, where the mirror is at rest in space of frame.
\begin{equation*}
    \Delta t^B = \frac{2d}{c} 
\end{equation*}

Consider now the same event in another frame $S_A$, where the Observer B and the mirror are moving with speed $v_B^A$. The path traced by light is clearly longer in $S_A$ than in $S_B$. According to Einstein's second postulate, however, light always propagates with the same speed in both frames. So, light clearly needs more time to cover the path in the frame $S_A$ than in $S_B$, and therefore time interval will be bigger.
\begin{equation*}
    \left(\frac{c\Delta t^a}{2}\right)^2 = d^2 + \left(\frac{v_b^a \Delta t^a}{2}\right)^2
\end{equation*}
After some derivation,
\begin{equation*}
    \Delta t^A = \frac{\Delta t^B}{\sqrt{1-\beta^2}} = \gamma \Delta t^B
\end{equation*}

\section{Längenkontraktion (Length Contraction)}
Length contraction is the phenomenon that is closely bounded with time dilation. The length of an object in the inertial system where it is located is called eigenlänge. In all inertial systems, where the object moves parallel to its length, is the measured length shorter than eigenlänge.
The length of rod in the frame $S_A$ when the rod is in other frame that moves relative to the named frame is \( l^{(A)} = x_2^{(A)} - x_1^{(A)}\). Inverse Lorentz-Transformation is used to determine positions of $x^{(A)}$ in endpoints of the time interval.
\[ l_{eigen} = x_2^{(B)} - x_1^{(B)} = \gamma (x_2^{(A)} - x_1^{(A)})\]
Solving for $x_2^{A} - x_1^{(A)}$ yields \(l = x_2^{(A)} - x_1^{(A)} = \frac{1}{\gamma} (x_2^{(B)} - x_1^{(B)})\)
\thm{Längekontraktion (Length Contraction) (or Lorentz-FitzGerald Contraction)}{
\begin{equation}
    l = \frac{1}{\gamma} l_{eigen} = l_{eigen} \sqrt{1-\beta^2}
\end{equation}
}
\subsection{Zerfallen der Myonen}
Myons decay by the following law:
\[N(t) = N_0 \exp{\frac{-t}{\tau}}\]
where timelife $\tau$ is around $2.2 \mu s$. However, since it has a speed of $\beta = 0.9978$, its timelife on Earth would be $33 \mu s$, so it covers 10 000m instead of 660 in frame of Earth.
\section{Doppler-Effect of Light (Relativistischer Doppler-Effekt)}
For the light and other electromagnetic waves in vacuum it gives no difference whether the source and the observer move relative to each other or not. Therefore, Doppler-Effect defined for waves earlier cannot account for the light, because the time intervals for the observer and the source were identical earlier, while it is not the case here.

Examine the source that moves with relative speed $v_b^a$ in the direction of a observer. The source emits in time interval $\Delta t^{(a)}$, which is measured in the frame of observer, n wave crests (Wellenberge) of EM wave. While the first wave crest in this time interval covers the line $c\Delta t^{(a)}$, the source moves covering the line of $v_b^a \Delta t^{(a)}$ towards the observer, measured in frame of observer. The wavelength of waves received by observer is therefore in this frame given by:
\[ \lambda^{(a)} = \frac{c\Delta t^{(a)} - v_b^a \Delta t^{(a)} }{n}\]
And the frequency is:
\[ \nu^{(a)} = \frac{c}{\lambda^{(a)}} = \frac{1}{1-\beta} \frac{n}{\Delta t^{(a)}}\]
Is the frequency of wave in the frame at rest (inertial?) of source equal to $\nu^{(b)}$, then the source emits in time interval $\Delta t^{(b)}$ \( n = v_b^a \Delta t^{(b)}\) wave crests. Then,
\begin{equation}
    \nu^{(a)} =  \frac{1}{1-\beta} \frac{v_b^a \Delta t^{(b)}}{\Delta t^{(a)}} = \sqrt{\frac{1+\beta}{1-\beta}}\nu^{(b)}
\end{equation}
for the diminishing distance. Such that, the frequency in frame of observer seems bigger than in source frame. On the contrary, if the source moves away from the observer, frequency will seem lower with the next mathematical expression:
\begin{equation}
    \nu^{(a)} = \sqrt{\frac{1-\beta}{1+\beta}}\nu^{(b)}
\end{equation}

An example of the relativistic Doppler-Effect is the red-shift of light that reaches us from the far galaxies.
\section{Clock synchronisation \& Simultaneity (Uhrensynchronisation und Gleichzeitigkeit)}

In the previous section it's seen that the eigenzeit (proper time) is the time interbal between two events in the frame where it is at rest / occurs on the same place. Therefore, proper time can be measured by a single clock. However, in other frame the events occur at two different points, and therefore two clocks are needed to measure the times of two events. However, this should require the clocks to be synchronised.

\thm{Synchronisation of clocks}{Two watches that are synchronised in one frame are typically not synchronous in another frame, that is moving relative to the first one.}
From this the following fact is derived related to the simultaneity of events:
\thm{Simultaneity of events}{Two events that are simultaneous in one frame, cannot be simultaneous in other frames, which are moving relative to the first frame.}

An exception is the case when the x-coordinates of both events are equal and the x-axis is parallel to the direction of relativistic motion of both frames.

How to synchronise two watches in frame $S_A$ that at distance l at points $x_1$ and $x_2$? We need to account for time travel of light $\frac{l^a}{c}$. 

\thm{Simultaneity (Gleichzeitigkeit)}{Two events occur simultaneously in a frame, if the light signals sent from the events reach the observer, who is located in the middle between two events, at the same time}

To show that two events that are simultaneous in frame $S_a$, aren't simultaneous in other frame $S_b$ that moves relative to first frame, it's needed to examine example proposed by Einstein. A train travels at constant speed $v_B^A$ past the train station. Train is at rest in frame B, station in frame A. Three observers are on the start, middle and end of train station (A) and train (B), and when two lightning bolts hits the train simultaneously, middle observer (A) sees the bolts at the same time. However, middle observer (B) sees the frontal bolt first.

\thm{Vorsprung der führenden Uhr}{
Will two clocks be synced in their Rest-frames, will they not be synced in other frames. In the frame where clocks are moving along the line between two clocks, the leading clock will be in front by a following contribution:
\begin{align}
    t_2^a - t_1^a &= \frac{v_b^a}{c^2} (x_2^a - x_1^a) \\
    \Delta t &= \frac{v_b^a}{c^2} \Delta x_{eigen}
\end{align}
}
\subsection{Das Zwillingsparadoxon}
Who will be older after the travel of Odysseus to Planet P and back to Earth? Homer or Odysseus? Answer: Homer will be (20-12) = 8 years older due to length contraction at $\beta = 0.8$. 

\section{Velocity transformation (Geschwindigkeitstransformation)}
The velocity particle in frame $S_A$ is \[v_x^a = \frac{dx^a}{dt^a}\]

From the Lorentz-transformation it follows that
\[ dx^a = \gamma (dx^b + v_b^a dt^a\]
and
\[ dt^a = \gamma \left( dt^b + \frac{v_b^a dx^b}{c^2}\right)\]

\thm{Relativistic Velocity-Transformation}{
\begin{align}
    v_x^{(A)} =& \frac{v_x^{(B)}+ v_B^{(A)}}{1+v_B^{(A)} v_x^{B}/c^} \\
    v_y^{(A)} =& \frac{v_y^{(B)}}{\gamma(1+v_B^{(A)} v_x^{B}/c^2)}\\
    v_z^{(A)} =& \frac{v_z^{(B)}}{\gamma(1+v_B^{(A)} v_x^{B}/c^2)}
\end{align}
The inverse transformations:
\begin{align}
    v_x^{(B)} =& \frac{v_x^{(A)}- v_B^{(A)}}{1-v_B^{(A)}v_x^{A}/c^2} \\
    v_y^{(B)} =& \frac{v_y^{(A)}}{\gamma(1-v_B^{(A)} v_x^{A}/c^2)}\\
    v_z^{(B)} =& \frac{v_z^{(A)}}{\gamma(1-v_B^{(A)} v_x^{A}/c^2)}
\end{align}}


\section{Relativistic Momentum (Relativistischer Impuls)}

In the classical mechanics the momentum of a body is $p = mv$. Relativistic momentum approaches $mv$ when v/c is approaching zero:
\thm{Relativistic Momentum}{
\begin{equation}
    p = \frac{mv}{\sqrt{1 - \beta^2} = \gamma mv}
\end{equation}
}
Examine in a simple Gedankenexperiment two observers: A in frame $S_A$ and B in $S_B$, which is moving in direction of +x-axis relative to frame $S_A$ with speed $v_B^{(A)}$. Both observers have hammers of mass m that are identical at rest. Observer B throws (schleudert) his hammer with velocity $-v = v_{2, y}^{(B)}$ in -y-axis direction. Observer A throws his hammer with velocity $v = v_{1, y}^{(A)}$ in +y-direction, such that both hammers collide elastically and return to observers.

\section{Relativistic Energy (die rel. Energie)}
In classical Mechanics, work done by resulting force on the body is equal to the change of kinetic energy of the body. In relativistic Mechanics resulting force is identified with the time change of rel. momentums. So we can calculate work done due to force that we equate with change of kinetic energy
\begin{align*}
    E_{kin} &= \int_{v=0}^{v=v_E} F ds =\int_0^{v_E} \frac{dp}{dt} ds = \int_0^{v_E} v dp \\ &= \int_0^{v_E} v d\left( \frac{mv}{\sqrt{1 - \beta^2}} \right),
\end{align*}
whereby \textit{v = ds/dt} is used.
\begin{equation*}
    d \left( \frac{mv}{\sqrt{1 - (v/c)^2}}\right) = m \left( 1 - \frac{v^2}{c^2} \right)^{-3/2} dv
\end{equation*}

\thm{Relativistic kin. Energy}{
\begin{equation}
    E_{kin} = \frac{mc^2}{\sqrt{1 - \beta^2} - mc^2} = mc^2(\gamma - 1)
\end{equation}
}

\thm{Rest Energy (Ruheenergie)}{ the rest energy is the product of rest mass and $c^2$
\begin{equation} \label{eq:rest_energy}
    E_0 = mc^2
\end{equation}
}
the relativistic (total-)energy is the sum of kinetic energy and rest energy:

\thm{Relativistic Energy}{
\begin{equation}
    E = E_{kin} + E_0 = \gamma mc^2
\end{equation}
}
This suggests that energy of body becomes very large when v approaches light speed. This can be interpreted as it is needed an infinite amount of energy to accelerate body with mass to light speed.

We can express velocity in terms of rel. momentum:
\begin{equation*}
    pc^2 = \gamma mc^2 v = Ev
\end{equation*}
or 
\begin{equation}
    \frac{v}{c} = \beta = \frac{pc}{E}
\end{equation}

In Atom- and Nuclear physics energies are usually given in Electronvolts (eV) or Megaelectronvolts (MeV):
\begin{equation*}
    1 \textrm{eV} = 1.602 \cdot 10^{-19} J 
\end{equation*}

For v \textless \textless c:
\begin{equation*}
    E_kin = mc^2 (1 + \frac{v^2}{2c^2} - 1) = \frac{1}{2} mv^2
\end{equation*}

\thm{Relationship between total energy, momentum, and rest energy}{
\begin{equation}
    E^2 = p^2c^2 + (mc^2)^2
\end{equation}
}

\section{Minkowski Diagramme}
\section{General Relativity Theory (allgem. Rel-theorie)}
The fundament of general relativity theory is the equivalency-principle:
\thm{Äquivalenzprinzip}{A homogeneous gravitational field is fully equivalent to equally accelerated frame}

Gravitations red-shift - shift to lower frequency (higher wavelength). That is due to light frequency being lower in region with low grav. potential (near Sun) than in a region with higher potential (near Earth).

\thm{Schwarzschild-Radius}{
The critical radius of black hole (when the object cannot longer escape the black hole) is given by:
\begin{equation}
    r_s = 2 \Gamma \frac{m}{c^2}
\end{equation}
}

\chapter{Quantum Mechanics}

\part{Introduction to Quantum Mechanics}
\section{Waves and Particles (Wellen und Teilchen)}
This section starts discussion of wave- and particle properties of EM waves and light in particular, with photoelectric effect at the top of table. 

The light that falls on a cathode, pushes the electrons from it, that settle in anode later. Thereby flows an electric current, which is measured by a amperemeter. 

The surprising result is that maximal electrical energy is independent of intensity of light that falls on cathode. Einstein explains it by postulating that light energy is quantised, so it enters in photons. A light beam consists of particles, photons, and the intensity is equal to the number of photons per unit area and time, multiplied by energy per photon. For the energy of photon the Einstein's equation passes:

\thm{Einstein'sche Gleichung für die Photonenergie}{
The energy E of a photon is:
\begin{equation}
    E = h \ \nu = \frac{ h \ c}{\lambda}
\end{equation}
Here $\nu$ is the frequency of light and h is Planck constant (Plank'sche Wirkungsquantum).
}

Interaction of light beams with metal surface consists in photoelectric effect in collision of photons and electrons. Thereby photons can be absorbed, whereby each photons gives his entire energy to an electron. So electron will be emitted from the surface after he obtains his kinetic energy from a photon that doesn't exist afterwards, with the same maximal kinetic energy given independent from intensity of light.

\thm{Einstein'sche photoelektrische Gleichung}{
\begin{equation}
    E_{kin, max} = (\frac{1}{2}m\ v^2)_{max} = h \ \nu - W_{ablöse}
\end{equation}
Where $W_{ablöse}$ is so-called Work that must be applied to emit electron from the surface of metal.
\begin{equation}
    W_{abl.} = h \nu_k = \frac{hc}{\lambda_k}
\end{equation}
where quantities with k subscript are critical/boundary frequency and wavelength.
}

\subsection{Compton-Scattering (-Streuung)}
The entire energy of photon doesn't necessarily transfer to electron, but they can scatter around free electrons (X-rays) and their frequency will be lowered.

Energy and momentum of classical EMW depend by
\begin{equation}
    E = p \ c
\end{equation}

So the conclusion for relation between momentum and wavelength of a photon is
\thm{Momentum of a photon}{
\begin{equation}
    p = \frac{E}{c} = \frac{h \ \nu}{c} = \frac{h}{\lambda}
\end{equation}
}

\thm{Compton Equation}{
\begin{equation}
    \lambda_2 - \lambda_1 = \frac{h}{m_e c} (1 - \cos\theta)
\end{equation}
}

The increase of wavelength therefore does not depend on the wavelength $\lambda_1$ of falling photon. 

Also, there is Compton-wavelength that is a part of respective equation:
\begin{equation}
    \lambda_{Compton} = \frac{h}{m_e c} = 2.426 \cdot 10^{-12} m = 2.426 pm
\end{equation}

\section{Particles as material waves (Teilchen als Materiewellen)}
As light was properties of waves and particles, there is a guess that matter  like electrons or protons also possess such characteristics.

For the wavelength of electrons sets de Broglie the expression $\lambda = h / p$:

\thm{De-Broglie equation}{
\begin{equation}
    \lambda = \frac{h}{p}
\end{equation}
where p is the momentum of electron.
}

Keep in mind that this relation corresponds for the photon. For the frequency of electron waves de Broglie chose Einstein's Equation, that adds together the frequency and energy of a photon:
\thm{De Broglie's Equation frequency of electron waves}{
\begin{equation}
    \nu = \frac{E}{h}
\end{equation}
}

In other microscopic elements the situation looks a little bit different:
\thm{De-Broglie wavelength of element}{
\begin{equation}
    \lambda = \frac{h}{p} = \frac{h}{\sqrt{2 \ m \ E_{kin}}} = \frac{h \ c}{\sqrt{2 \ m \ c^2 E_{kin}}}
\end{equation}
}

Where we again have the value $h \ c = 1240 eV\cdot nm$ and $m c^2 = 0.5110 MeV$ for electrons.
\begin{equation*}
    \lambda = \frac{1240 eV\cdot nm}{\sqrt{2 (0.5110 \cdot 10^6 eV) E_{kin}}}
\end{equation*}
\thm{Wavelength of an electron}{If the energy of electron is given in eV, its wavelength can be calculated by following:
\begin{equation}
    \lambda = \frac{1.226}{\sqrt{E_{kin}}} \ nm
\end{equation}
}
These equations do not count for relativistic elements, whose kinetic energy make up notable fraction of their rest energy.


\sometitle{Interference and Diffraction of electrons}

This is well used in electron-microscopes to get a picture of microscopic elements that have a wavelength smaller than of a visible light.

\section{Schrödinger Equation/Gleichung}
Wave function of waves - deviation y in dependence of place and time, wave function of sound - deviation s of air molecules or the pressure p, and wave function of EM waves is the electrical field and magnetic field vectors E and B.

Wave function of electrons waves ist labelled with $\psi$ which is a partial differential equation in space and time. 
\thm{Time-dependent Schrödinger Equation}{In one dimension is it:
\begin{equation}
    -\frac{\hbar^2}{2m} \frac{\partial^2 \psi(x, t)}{\partial x^2} + E_{pot} \ \psi (x,t) = i \hbar \frac{\partial \psi (x,t)}{\partial t}
\end{equation}
}
The function $\psi (x, \ t)$ is not necessarily directly measurable since it can have imaginary solution due to presence of imaginary number i.

An answer how to interpret that is given by quantisation of Light waves. Shortly, it can be assumed that there is probably to find either 1 or 0 photons per unit volume.

\begin{equation}
    P (x, \ t) dx = |\psi (x, \ t) |^2 dx = \psi^{*} \psi dx
\end{equation}
whereby $\psi^*$ is complex conjugated function to $\psi$, and their product is real number.

\thm{Aufenthaltswahrscheinlichkeit}{
\begin{equation}
    P (x, \ t) dx = |\psi (x, \ t) |^2 dx = \psi^{*} (x, \ t ) \psi (x, \ t) dx
\end{equation}}

Equation for one-dimensional can be expressed by following:
\begin{equation}
    \psi (x, \ t) = \psi (x) e^{-i \omega t}
\end{equation}
where, $e^{-i\omega t} = \cos (\omega t) - i\sin (\omega t)$.

So, we get time independent Schrödinger equation:
\thm{Time-independent Schrödinger Equation}{
\begin{equation}
     -\frac{\hbar^2}{2m} \frac{\partial^2 \psi(x, t)}{\partial x^2} + E_{pot} \ \psi (x,t) = E \psi (x)
\end{equation}
Where, $E = \hbar \omega$ is the energy of a particle.}

\thm{Normierungsbedingung (Uslovie normirovanija)}{
When a particle is present, there must be a probability of 1 to find it somewhere:
\begin{equation}
    \int^{+\infty}_{-\infty} |\psi|^2 dx = \int^{+\infty}_{-\infty} \psi^{*} (x) \psi (x) dx = 1
\end{equation}
}

\section{Wave-Particle Dualism}
Electrons that is often regarded as particles, can show wave properties when it passes slits, showing interference and diffraction effect. A classical particle behaves like a winzige massive sphere: it is localized, it can be scattered, and it transfers energy with itself, which it can exchange upon collision in a point in space in a definite time. However, it doesn't show any interference or diffraction effect. A classical wave, on the other hand, behaves like a sound or light wave: it shows interference and a diffraction, and it's energy is distributed evenly in space and time continuously. 

\sometitle{Das Doppelspaltexperiment}

When the distance between slits is equal odd multiple of the half wavelengths, the wave function $\psi$ is equal to zero, and the electron will meet with very low probability to hit screen near the minima. When the distance between the slits is an even multiple of wavelengths, $\psi$ is maximum, and electron has a high probability to hit screen near this maxima, with a probability of $\psi^2$.

\sometitle{Heisenberg's Uncertainty Principle (Unschärferelation)}

Es means that is it principally impossible to simultaneously measure with any precision/accuracy the position as well as the momentum of a particle. Normally, a position of an object is obtained with the help of light. 

\thm{Heisenberg's Uncertainty Principle}{
Taken that $\Delta x$ and $\Delta p$ are standard deviation in measuring of position and momentum, then it can be shown that their product must be at least equal $\hbar / 2$:
\begin{equation}
    \Delta x\Delta p \> \frac{\hbar}{2}
\end{equation}
Here, $\hbar = \frac{h}{2\pi}$
}

\section{Erwartungswerte und Klassischer Grenzfall}
The average value of x that from measuring positions of numerous particles with the same wave function $\psi (x)$, labelled as $\langle x \rangle$ is the expected value (Erwartungswert).

\thm{Definition of expected value of x and function F(x)}{
\begin{align}
    \langle x \rangle &= \int_{-\infty}^{+infty} x |\psi(x)|^2 dx \\
    \langle F(x) \rangle &= \int_{-\infty}^{+infty} F(x) |\psi(x)|^2 dx
\end{align}
}

\subsection*{Klassicher Grenzfall}

\thm{Ehrenfest-Theorem (Korrespondenzprinzip)}{
The expected values of quantum mechanical quantities fulfill approximately the same motion equations as the equivalent classical quantities.
}

\thm{Bohr's correspondence principle}{
In border-case of very high quantum numbers the classical and quantum calculations must bring the same result.
}
\part{Application of Schrödinger Equations}
\section{A particle in a box with infinitely high potential}
\section{A particle in a box with finitely high potential}
\section{Harmonic Oscillator}
\section{Reflection and Transmission of electron waves on potential barriers}
\section{Schrödinger Equation in 3D}
\section{Schrödinger Equation for 2 identical particles}

\chapter{Atoms and Molecules}
\part{Atoms}
Dosele byli obnaruženy 118 elementov chymicznych, z których 92 naideny v prirode. Kožny element vydeljaetsja czislom protonov Z i elektronov i czislom N neitronov v atome. 
\section{Atom and Atomic Spectre}
Light radiated by different atoms can be observed with a spectroscope to have specific lines of different colors or different wavelengths. Reciprocal wavelength can be calculated with Rydberg-Ritz-Equation:
\begin{equation}
    \frac{1}{\lambda} = R \left(\frac{1}{n_2^2} - \frac{1}{n_1^2} \right)
\end{equation}
Where R is the Rydberg constant and $n_1 \> n_2 $ and integers.
\thm{Rydberg constant of Hydrogen atom}{
For Hydrogen, Rydberg constant has a value of:
\[ R_H = 1.097 \ 373 \ 156 \ 8160(21) \cdot 10^7 \textrm{m}^{-1}\]
}

This equation produces reciprocal wavelengths for spectral lines of hydrogen and alkali metals like lithium and natrium. 

Thomson's Model of Atom - Pudding model. Classical theory that implied that electrons are scattered in something like a liquid that contains a bigger part of atom mass and was very positively charged to make atom neutral. Big problem with it was due to explaining interaction of charges in electromagnetism.

It was later disproven by experiment of Ernest Rutherford and his apprentices, Hans Geiger and Ernest Marsden. In gold foil experiment, they shot alpha-particles onto a gold foil and observed that most particles deviated with big angle, and very few didn't deviate. This also implied that positive charge and mass of atom to be concentrated in a very small region of atom (nucleus), around $1 fm = 10^{-6} nm$.

\section{Bohr's Model of Hydrogen Atom}
Niels Bohr postulated a model in 1912 that could explain the observed spectres. In his model, an electron circles around a positively charged nucleus in a circular or elliptical path, where forces of Coulomb and laws of classical physics pass. 

Potential energy of electron at distance r from positive charge $+Z\ e$ is
\begin{equation}
    E_{pot} = \frac{1}{4\pi \varepsilon_0} \frac{q_1 q_2}{r} = -\frac{1}{4\pi \varepsilon_0} \frac{Z \ e^2}{r}
\end{equation}
Where $4 \pi \varepsilon_0$ is Coulomb's constant. The kinetic energy of electron is derived from Newton's second law $F = ma$:
\begin{equation}
    E_{kin} = \frac{1}{2} m v^2 = \frac{1}{8\pi \varepsilon_0} \frac{Ze^2}{r}
\end{equation}

Kinetic energy and potential energy are therefore inversely proportional to path radius r. Contribution of potential energy is 2 times bigger than of kinetic energy:
\begin{equation}
    E_{pot} = -2E_{kin}
\end{equation}
Total energy is the sum of kinetic and potential energy:
\[E = E_{kin} + E_{pot} = \frac{1}{8\pi \varepsilon_0} \frac{Ze^2}{r} -\frac{1}{4\pi \varepsilon_0} \frac{Z \ e^2}{r}\]
So we can formulate the expression for the energy on a circular path by electrostatic force:
\thm{Energy on a circular path by electrostatic $1/r^2$ force}{
\begin{equation}
    E = - \frac{1}{8\pi \varepsilon_0} \frac{Ze^2}{r}
\end{equation}
}

In order to solve difficulties of this model with the laws of electrodynamics, Bohr developed some postulates, of them being that only confirmed circular paths are allowed and atom doesn't radiate with an electron that circles around nucleus. He proposed paths to be stationary states, where electrons cannot radiate during moving.

\thm{First Bohr's Postulate}{
The electron in hydrogen atom can be moving without radiation only on circular, defined paths (stationary states).
}

The second Bohr's postulate connects the frequency of spectral radiation with the energies of divided stationary states
\thm{Second Bohr's Postulate}{
If $E_A$ is initial and $E_E$ the final energy of atom, then frequency of radiation emitted in a transition given by:
\begin{equation}
    \nu = \frac{E_A - E_E}{h}
\end{equation}
}

\thm{Third Bohr's Postulate: Quantised Rotational Momentum}{
The rotational impulse of a particle in a circular path is $m v r$. So:
\begin{equation}
    m v_n r_n = \frac{nh}{2\pi} = n \hbar, \quad n = 1, 2, 3, \dots
\end{equation}

\dfn{Radius of Bohr's Electron paths}{
For the radius of Bohr's electron paths passes:
\begin{equation}
    r_n = n^2 (4\pi \varepsilon_0) \frac{\hbar^2}{m Z e^2} = \frac{n^2 a_0}{Z}
\end{equation}
}
Where $a_0 = 0.0529 nm$ is the first Bohr's radius.
}
\sometitle{Energy levels of Hydrogen atom}
The total mechanical energy of electron in hydrogen atom depends on radius of its circular path. If we insert quantised values of $r_n$, we obtain:
\begin{align*}
    E_n &= - \frac{1}{8\pi \varepsilon_0} \frac{Z e^2}{r_n} =  - \frac{1}{8\pi \varepsilon_0} \frac{Z^2 e^2}{n^2 a_0} \\ 
    &= - \frac{1}{2} \frac{1}{(4\pi \varepsilon_0)^2} \frac{m Z^2 e^2}{n^2 \hbar^2}
\end{align*}
Or the energy levels of hydrogen atom.
\thm{Energy level of Hydrogen Atom}{
\begin{equation}
    E_n = -Z^2 \frac{E_0}{n^2}
\end{equation}
}

Transitions between these allowed energy levels is linked with the emission or absorption of a photon, whose frequency is given by $\nu = \frac{|E_A - E_E|}{h}$. Respectively, its wavelength is:
\begin{equation}
    \lambda = \frac{c}{\nu} = \frac{hc}{ |E_A - E_E|}
\end{equation}

\section{Quantum theory of atoms}
Quantum mechanics describes the electron by a wave function (probability to be found in some volume element dV). Schrödinger equation for it is
\begin{equation}
    - \frac{\hbar^2}{2m}\left( \frac{\partial^2 \psi}{dx^2} + \frac{\partial^2 \psi}{dy^2} + \frac{\partial^2 \psi}{dz^2}\right) + E_{pot}(x, y, z)\psi = E\psi
\end{equation}
In a single isolated atom, potential energy depends only on radial distance $r = \sqrt{x^2+y^2+z^2}$ of an electron from the middle of atom nucleus. In the further calculation it's best to use polar coordinates r, $\theta$, and $\phi$:
\begin{align*}
    x &= r \sin\theta \cos\phi \\
    y &= r \sin\theta \sin\phi \\
    z &= r \cos\theta 
\end{align*}

After the transformation to polar coordinates, the first step is to solve a partial differential equation by separation of variables:
\begin{equation}
    \psi (r, \theta, \phi) = R(r) f(\theta) g(\phi)
\end{equation}

In 3 dimension rules the requirement that the wave function is stetig and normalisable to 3 quantum numbers that are linked with each of space dimension. In polar, r is linked with n, $\theta$ with $\ell$ and $\phi$ with $m_{\ell}$.

\thm{Quantum numbers in polar coordinates}{
They can accept following values:
\begin{align}
    n &= 1, 2, 3, \dots \\
    \ell &= 0, 1, 2, 3, \dots, n - 1 \\
    m_{\ell} &= -\ell, (-\ell + 1), \dots, \ell
\end{align}
}
Number n is called main quantum number. $\ell$ is called quantum number of rotational momentum. The contribution L of path's rotational momentum is therefore:
\begin{equation}
    L = \sqrt{\ell(\ell+1)}\hbar
\end{equation}

\section{Quantum theory of hydrogen atom}
\section{Spin-Bahn Kopplung}
\section{Period system of elements}
\section{Spectres in visible light and in rentgen regions}
\section{Laser}

\part{Molecules}
This chapter is fully optional and is outside of university volume 3. It is however present in Tipler, so it is included here. In future, i might expand it with material from Gerthsen Physics.
\section{Chemical Binding}
\section{Polyatomic Molecules}
\section{Energy states and Spectres in two-atomic molecules}
\section{Degree of liberty and Law of even distribution}



\part{Condensed Matter Physics (Festkörper)}
\part{Atom- and Nuclear Physics}

\end{document}
